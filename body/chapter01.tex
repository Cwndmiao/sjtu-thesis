%%==========================
%% chapter01.tex for SJTU Master Thesis
%% based on CASthesis
%% modified by wei.jianwen@gmail.com
%% version: 0.3a
%% Encoding: UTF-8
%% last update: Dec 5th, 2010
%%==================================================

%\bibliographystyle{sjtu2} %[此处用于每章都生产参考文献]
\chapter{绪论}
\label{chap:what}

\section{引言}

人类对计算资源的需求永远是无止境的。

在过去的半个多世纪里,信息技术领域的飞速发展,尤其是计算机不断增强的计算性能和互联网技术方面的进步极大地改变了人们的生活和工作方式。而云计算则采用创新的计算模式使用户能够通过网络随时获得近乎无限的计算能力和丰富多样的信息服务,它同时具备创新的商业模式,使用户对计算和服务可以自由、按量付费,极大地增强了给用户带来的可选择性和便利性。

现有的云计算平台可以按照不同的服务层次分类,设施基础即服务(Infrastructure as a Service,IaaS)是其中尤为重要的一种,即云服务提供商提供给用户的计算资源为包括处理器、内存、磁盘等在内的抽象硬件。而用户则负责在抽象硬件上配置安装操作系统和上层应用程序。在IaaS这种服务架构下用户可以获得极大的自由度,因为用户能够自由配置抽象硬件之上的整个软件栈。美国亚马逊公司在2006开始发布的弹性云计算(Amazon Elastic Compute Cloud,Amazon EC2)系统便是IaaS架构中的典型代表。用户可以根据应用的实际需要在EC2上弹性地选择要购买的抽象硬件配置,用自己的Amazon机器镜像文件创建虚拟机,并在这个虚拟机上运行任何自己想要的软件或应用程序。除了Amazon EC2,众多高科技互联网公司均在建设自家的类似IaaS云计算平台,如谷歌的GCE(Google Compute Engine)和微软的Azure。

一个典型IaaS架构的云计算平台由虚拟机监控器和若干在其上运行的客户虚拟机组成,而系统虚拟化技术则是IaaS架构能付诸实现的关键。

\section{系统虚拟化的优势}

系统虚拟化是计算机系统上众多虚拟化技术中的一种,相比于进程虚拟化、文件系统虚拟化等,其进行资源抽象的粒度为整个计算机。与传统的计算模式相比,依赖系统虚拟化技术在云计算平台上运行虚拟机具备如下优势:

\begin{itemize}
\item 统一简化管理操作,提高管理效率。
\item 整合冗余服务器,提高硬件利用率,降低资源消耗成本。
\item 允许旧版软件系统与新硬件环境共存,增强兼容性。
\item 提高服务可靠性,使用迁移技术可以将虚拟机从故障硬件转移。
\end{itemize}

\section{现有虚拟化平台上存在的问题}

然而,世上难有完美事物,尽管系统虚拟化技术随着云计算平台的迅猛兴起而得到了广泛运用,研究人员和开发社区在此领域中投入了大量精力,现有虚拟化平台之上仍然存在着不少亟待解决的问题。

虚拟机内的用户隐私数据存在暴露风险。虚拟机监控器处于系统最高权限级别运行,其在客户虚拟机运行过程中,可以随时窥探虚拟机内存和磁盘中的数据,对用户存储在虚拟机内的隐私数据构成威胁。

除此之外,在多虚拟机整合运行的云计算服务器上,并行应用程序的实际性能存在严重下降。这是由于,虚拟机监控器不清楚虚拟机内部的实际运行情况,导致其盲目做出了不合理的调度行为,延缓阻碍了虚拟机内部关键进程的进行。

\section{本文的主要贡献}

针对现有虚拟化平台上虚拟机运行存在的安全性和性能问题,本文对其进行了深入的剖析探讨,提出并实现了可行的解决方案。本文做出的主要贡献如下:

\begin{itemize}
\item 针对虚拟机的安全性问题,作者利用嵌套式虚拟化,在嵌套式虚拟化层加入了对虚拟机监控器的行为监控。在利用此方式实现的Secure KVM系统中,我们以可接受的性能损耗作为代价,增强了多租户云虚拟化平台上客户虚拟机运行的安全性。
\item 针对多虚拟机整合运行时的性能问题,我们实现了FlexCore系统,其在KVM平台上完整实现了vCPU ballooning调度方法。通过在虚拟机运行时动态调整赋予其的虚拟处理器数目,FlexCore主动规避因争抢物理核而造成的巨大性能损耗,提升虚拟机内应用程序的实际运行性能。
\end{itemize}

\section{本文结构安排}

本文的内容共分为五章。

第一章的绪论部分至此基本结束。

第二章介绍了x86系统虚拟化技术的发展历程,以及阅读后续章节所必备的一些x86硬件辅助虚拟化背景知识。

第三章介绍Secure KVM安全嵌套式虚拟化系统,该系统用于解决虚拟化平台上客户虚拟机运行时的安全问题。

第四章介绍FlexCore系统,该系统用于解决多虚拟机整合运行时的性能问题。

第五章对本文的研究工作和成果进行总结,并提出了未来进一步的研究方向。

最后,附录部分对本人在攻读硕士学位期间参与科研项目情况和发表论文情况进行了总结,并致谢。



