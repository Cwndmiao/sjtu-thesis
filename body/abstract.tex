%%==================================================
%% abstract.tex for SJTU Master Thesis
%% based on CASthesis
%% modified by wei.jianwen@gmail.com
%% version: 0.3a
%% Encoding: UTF-8
%% last update: Dec 5th, 2010
%%==================================================

\begin{abstract}

云计算是一种新型计算模式,其基于网络可以向用户提供按需定制的、可扩展的计算资源,最终可以让用户像使用水电等基础设施一样方便地使用计算资源。近年来,云计算逐渐成为工业界重要的发展趋势,同时也成为学术界研究的热点。而系统虚拟化则是云计算实现中的关键技术。借助系统虚拟化,为用户提供类似于真实硬件的虚拟机抽象,不仅可以帮助云服务提供商减少服务器数量、优化资源利用率,还可以帮助用户实现弹性基础设施配置,从而降低成本、快速响应业务需求变化。当前主要的云服务提供商如Amazon,Microsoft,Google等通常以虚拟机为单位为客户提供计算资源。

随着处理器硬件虚拟化特性的加入以及运算能力的不断提高,系统虚拟化在实际场景中得到了广泛应用,例如多租户云、虚拟机聚合等。然而,事物总有其双面性,虚拟化平台尽管带来了便利,其上仍然存在着不少亟待解决的问题。

虚拟机内用户隐私数据的安全问题是其中较为突出的一个。在现有虚拟化平台中,虚拟机监控器处于系统最高权限运行,其能够接触到客户虚拟机运行时所有的内存和磁盘数据,并任意访问其中的用户隐私,给云平台上的数据安全造成威胁。除此之外,由于虚拟机监控器和客户虚拟机之间存在语义隔阂,虚拟机监控器做出的盲目调度行为会使得并行应用程序在多虚拟机整合运行情形下的性能严重降低。如何提升虚拟机整合运行时的系统运行效率,是另一个颇有实际意义的研究问题。

本文针对上述虚拟化环境中安全性和性能方面的两个问题,深入剖析了其内在成因,提出并实现了可行的解决方案,修改或重新设计了现有系统,改善了虚拟化环境中操作系统的安全性和性能。

具体而言,本文的主要贡献可以概括如下:

\begin{itemize}
\item 基于嵌套式虚拟化实现了Secure KVM系统。通过在原虚拟机监控器KVM下方添加安全嵌套式虚拟化层,Secure KVM将原先占据系统最高权限的KVM调整到非根模式运行,并在嵌套式虚拟化层中监视KVM做出的所有与客户虚拟机相关的操作。Secure KVM在保障了客户虚拟机正常运行的同时,避免其中的用户隐私数据遭到恶意虚拟机监控器窥探,以可以接受的性能开销增强了云环境中的数据安全。
\item 在KVM虚拟化平台上完善并实现了vCPU Ballooning虚拟机调度算法。vCPU Ballooning是宋翔博士首先在Apsys'13提出的一种全新虚拟机调度策略,其通过主动减少赋予虚拟机的vCPU数目来避免因多个vCPU竞争物理处理器而造成的性能损耗。本文在该策略初步构想的基础上,细致剖析了该策略奏效的内在原因,完善了具体的调度算法,并在KVM平台上给出了完整实现。测试结果表明,该调度算法给并行应用程序测试集带来了高达50\%的平均性能提升。
\end{itemize}

在本文最后,作者还就解决上述问题的可能新途径和系统虚拟化领域的未来发展方向提出了展望。

  \keywords{\large 系统虚拟化 \quad 隐私安全 \quad 嵌套式虚拟化 \quad 虚拟机聚合 \quad vCPU Ballooning}
\end{abstract}

\begin{englishabstract}

In recent years, because of the ability of cloud computing to provide scalable, highly efficient virtual computing resources under the pay-as-you-go model, cloud computing has gained significant momentum in both academia and industry. Virtualization, a key enabling technology for cloud computing, facilitates efficient resource allocation to end users in the form of virtual machines. With system virtualization, cloud service providers are able to optimize resource utilization, thus reduce cost. End users also gain elastic infrastructure configuration, fulfilling business requirements on demand. Amazon and Google both deliver their services in IaaS model.

With hardware virtualization support added to modern x86 CPU, as well as its growing capacity, virtualization is widely adopted in real-world scenarios, viz., multi-tenant cloud and server consolidation. However, many urgent problems still exist on top of current virtualization platforms. 

How to protect data privacy and integrity in guest VM is of vital importance. In most cases, virtual machine monitor runs with the highest privilege in system, thus, it's able to monitor and access arbitrary memory and disk data of guest VM. In case of malicious VMM or cloud operator, security-sensitive data in guest VM is prone to be compromised, raising new security challenges in third-party multi-tenant cloud. Unfortunately, commodity virtualization infrastructures usually provide limited assurance for data security. 

Double scheduling may lead to severe performance degradation of SMP-VMs. In essence, this phenomenon is caused by inappropriate scheduling issued by the hypervisor because most operations inside a guest VM are opaque to the hypervisor, which is referred to as ``semantic gap''. However, commodity hypervisors do not consider carefully this performance issue caused by double scheduling. This leaves considerable room for optimization.

In order to address the problems mentioned above, thus improving the security and performance of guest VM in virtualized environments, this thesis proposes the Secure KVM system and the FlexCore system, via modifying or re-designing current ones. Specifically, the main contributions of this thesis consist of the following:

\begin{itemize}
\item Based on nested virtualization, the thesis presents the Secure KVM system, which inserts a nested virtualization layer and deprivileges the KVM hypervisor. As any VMX operation by the deprivileged KVM in non-root mode causes trap to the nested virtualization layer, Secure KVM is able to monitor all guest-VM-related activities by KVM, filtering insecure/invalid intents like mapping memory pages belonging to guest VM. With acceptable performance penalty, Secure KVM enhances data privacy and integrity in guest VM notably, according to the attacking and defending demos in Chapter 3.
\item To mitigate the huge performance degradation caused by double scheduling, the thesis incorporates vCPU Ballooning, viz., a novel VM scheduling strategy, into KVM and presents the FlexCore system. Chapter 4 first identifies function-call IPI delay and spinlock holder preemption as the two main causes that lead vCPUs entering long and meaningless busy waiting states. Unlike traditional approaches, FlexCore proactively avoids contention by reducing the number of vCPUs. Our evaluation shows that the average performance improvement for PARSEC applications is approximately 52.9\%.
\end{itemize}

In the end, the author also forecast several possible novel approaches and future research directions of system virtualization.


  \englishkeywords{\large Virtualization, Data privacy, Nested virtualization, VM consolidation, vCPU Ballooning}
\end{englishabstract}
