%%==================================================
%% abstract.tex for SJTU Master Thesis
%% based on CASthesis
%% modified by wei.jianwen@gmail.com
%% version: 0.3a
%% Encoding: UTF-8
%% last update: Dec 5th, 2010
%%==================================================

\begin{abstract}

TODO

云计算是一种新的基于互联网的计算方式,可以为客户提供按需供应的、可扩展的计算资源,最终可以使得用户像使用水电等基础设施一样方便的使用计算资源。近年来,云计算成为工业界重要的发展趋势,同时也成为学术界研究的热点。而虚拟化(virtualization)是云计算实现的关键技术。要实现云计算,首先必须从基础架构虚拟化开始。通过服务器虚拟化、网络虚拟化、应用虚拟化等解决方案,不仅可以帮助用户减少服务器数量、优化资源利用率,还可以帮助用户实现动态IT基础设施环境,从而降低成本、快速响应业务需求的变化。服务器虚拟化是目前虚拟化技术应用的重要领域。当前主要的云服务提供商如Amazon,Microsoft,IBM等通常以虚拟机为单位为客户提供计算资源。

随着硬件虚拟化能力的加入以及计算能力的不断提高,系统虚拟化在实际场景中得到了广泛运用,例如服务器聚合、多租户云等。利用虚拟化技术,多台运行不同服务的虚拟机可以被部署到同一台物理机上,以最大化资源利用率。由于物理机CPU的核心数在不断增加,为了充分利用具有的计算资源,具有多个VCPU的虚拟机开始普及。

然而虚拟机内部的运行状况对于下层虚拟机监控器基本上是透明的,这带来了在非虚拟化环境下不存在的双重调度的问题,虚拟机内部的OS需要把进程调度到VCPU上,而虚拟机监控器需要把VCPU调度到物理核上。于是,某个VCPU在拥有自旋锁时仍有可能被其他VCPU抢占,造成同步延迟的增加,而被调度上来的VCPU由于拿不到锁,只能白白浪费被分配到的时间片。双重调度的问题会造成虚拟机实际性能的下降,尤其是在运行大型并行程序时。如何调整虚拟机监控器的运行策略,提高系统整体的运行性能,是一个有着深远意义的问题。

除了以上提及的多个VCPU竞争一个物理核的问题,虚拟化平台为虚拟机提供硬件抽象(包括CPU和内存等),隐藏了真实的硬件细节,例如内存的抽象通常采用UMA模型,取消了真实NUMA拓扑结构。因此虽然现代操作系统从任务调度和内存分配等方面对NUMA硬件环境做出优化,并且取得了显著的性能提升。但是信息的不透明使得操作系统在虚拟化环境下无法利用这些优化方法,从而阻碍了虚拟机性能的提升。

同时,客户虚拟机的运行要依赖下层的虚拟机监控器为其提供服务,其内存和磁盘等可能存储机密数据的介质直接暴露在虚拟机监控器之下,虚拟机监控器可以不受限制地访问或修改客户虚拟机中的任何数据。如何在维护客户虚拟机正常运行的同时,保护其内存和磁盘数据的私密性和完整性,是近年来学术界讨论和研究的一个焦点。

本学位论文将围绕以上提及的两个问题,深入分析了解其中的根本原因,修改或者重新设计现有系统,提高客户虚拟机在虚拟化环境下的可伸缩性和安全性。


  \keywords{\large 上海交大 \quad 饮水思源 \quad 爱国荣校}
\end{abstract}

\begin{englishabstract}

TODO

云计算是一种新的基于互联网的计算方式,可以为客户提供按需供应的、可扩展的计算资源,最终可以使得用户像使用水电等基础设施一样方便的使用计算资源。近年来,云计算成为工业界重要的发展趋势,同时也成为学术界研究的热点。而虚拟化(virtualization)是云计算实现的关键技术。要实现云计算,首先必须从基础架构虚拟化开始。通过服务器虚拟化、网络虚拟化、应用虚拟化等解决方案,不仅可以帮助用户减少服务器数量、优化资源利用率,还可以帮助用户实现动态IT基础设施环境,从而降低成本、快速响应业务需求的变化。服务器虚拟化是目前虚拟化技术应用的重要领域。当前主要的云服务提供商如Amazon,Microsoft,IBM等通常以虚拟机为单位为客户提供计算资源。

随着硬件虚拟化能力的加入以及计算能力的不断提高,系统虚拟化在实际场景中得到了广泛运用,例如服务器聚合、多租户云等。利用虚拟化技术,多台运行不同服务的虚拟机可以被部署到同一台物理机上,以最大化资源利用率。由于物理机CPU的核心数在不断增加,为了充分利用具有的计算资源,具有多个VCPU的虚拟机开始普及。

然而虚拟机内部的运行状况对于下层虚拟机监控器基本上是透明的,这带来了在非虚拟化环境下不存在的双重调度的问题,虚拟机内部的OS需要把进程调度到VCPU上,而虚拟机监控器需要把VCPU调度到物理核上。于是,某个VCPU在拥有自旋锁时仍有可能被其他VCPU抢占,造成同步延迟的增加,而被调度上来的VCPU由于拿不到锁,只能白白浪费被分配到的时间片。双重调度的问题会造成虚拟机实际性能的下降,尤其是在运行大型并行程序时。如何调整虚拟机监控器的运行策略,提高系统整体的运行性能,是一个有着深远意义的问题。

除了以上提及的多个VCPU竞争一个物理核的问题,虚拟化平台为虚拟机提供硬件抽象(包括CPU和内存等),隐藏了真实的硬件细节,例如内存的抽象通常采用UMA模型,取消了真实NUMA拓扑结构。因此虽然现代操作系统从任务调度和内存分配等方面对NUMA硬件环境做出优化,并且取得了显著的性能提升。但是信息的不透明使得操作系统在虚拟化环境下无法利用这些优化方法,从而阻碍了虚拟机性能的提升。

同时,客户虚拟机的运行要依赖下层的虚拟机监控器为其提供服务,其内存和磁盘等可能存储机密数据的介质直接暴露在虚拟机监控器之下,虚拟机监控器可以不受限制地访问或修改客户虚拟机中的任何数据。如何在维护客户虚拟机正常运行的同时,保护其内存和磁盘数据的私密性和完整性,是近年来学术界讨论和研究的一个焦点。

本学位论文将围绕以上提及的两个问题,深入分析了解其中的根本原因,修改或者重新设计现有系统,提高客户虚拟机在虚拟化环境下的可伸缩性和安全性。

  \englishkeywords{\large SJTU, master thesis, XeTeX/LaTeX template}
\end{englishabstract}
