%%==================================================
%% chapter03.tex for SJTU Master Thesis
%% Encoding: UTF-8
%%==================================================

\chapter{Secure KVM安全嵌套式虚拟化系统的设计与实现}
\label{chap:securekvm}



\section{引言}

\section{背景介绍及问题分析}

\subsection{虚拟化本质论}

所谓“系统虚拟化”,亦即VMM虚拟机监控器给客户操作系统和应用程序提供一个虚拟的运行平台,而这个虚拟运行平台要与真实硬件基本无异,不能让客户操作系统和应用程序感受到差别(现阶段,打通VMM虚拟机监控器和Guest VM客户虚拟机之间的语义隔阂,主动让客户虚拟机意识到自己运行在虚拟化平台上以获得性能提升的情形也是存在的,在本文中我们忽略这样的特例)。

对于一个虚拟的运行平台而言,有三个主要组成要素,分别是CPU、内存和外设(磁盘和网卡属于此类)。关于CPU,虚拟机监控器要将物理机器上的计算资源即CPU时间片分一部分给客户虚拟机,同时又必须保证CPU资源不能被某一客户虚拟机长期无节制地占用,以免耽误虚拟机监控器自身和其他客户虚拟机的正常运行(类似于操作系统和应用程序之间的关系)。对于内存,虚拟机监控器同样要在满足客户虚拟机内存资源分配的同时保证安全,即某一客户虚拟机必须有节制地使用内存,该客户虚拟机和虚拟机监控器、该客户虚拟机和其他客户虚拟机之间必须保证内存隔离。对于外设,虚拟机监控器要对他们的IO行为进行精确模拟,这一般是通过截获IO指令和MMIO读写操作来完成。

这所有的一些,均要求虚拟机监控器处在比客户虚拟机更高的运行级别。对于CPU,虚拟机监控器要在某一客户虚拟机时间片用完时(时间中断到来),立即获得执行机会抢占该客户虚拟机,并切换另一客户虚拟机上来运行。对于内存,虚拟机监控器要通过一些类似页表的硬件机制限制客户虚拟机的访存范围。对于截获IO指令和MMIO读写,这同样需要高运行级别和硬件机制来保证。

具体到x86硬件虚拟化,虚拟机监控器处在高权限级别的根模式(root mode),客户虚拟机处在低权限级别的非根模式(non-root mode)。当客户虚拟机处在非根模式运行期间发生时间中断时,处理器会以外部中断(External Interrupt)到来的原因陷入到根模式,让虚拟机监控器处理。2007之后Intel推出的第二代硬件虚拟化处理器,均支持扩展页表(Extended Page Table,简称EPT),可以在硬件层面上限制客户虚拟机的内存访问。最后,虚拟机监控器可以指定IO位图,让非根模式下期望的IO指令发生陷入(MMIO读写与此基本类似),并对其进行指令模拟以达到模拟真实设备行为的目的。

不可否认,为了达到系统虚拟化的目的,在一般情况下让虚拟机监控器处在高权限级别是一个必要条件。但与此同时带来的是,虚拟机监控器不受管控的“为所欲为”,给客户虚拟机运行的安全性和隐私性带来了重大隐患。此情形在多租户云、第三方虚拟机提供商等应用环境下体现得尤为明显。下面两节分别从内存和磁盘的角度对此进行详细阐述。

\subsection{客户虚拟机的内存安全}

客户虚拟机的内存中包含有其正在运行程序的所有信息,包括操作系统层面的进程信息、模块信息和应用程序地址空间中的所有数据、代码区域。举例来说,客户虚拟机中的浏览器打开了网银的登录界面而用户正在输入其账号密码,这些隐私数据都是会被保存到客户虚拟机的内存中。

扩展页表限制的是客户虚拟机的访存范围,而虚拟机监控器处在最高权限级别,其可以将物理上的任意内存区块映射到自己的地址空间,并进行访问改写。同样以上文用户网银登陆为例,恶意虚拟机监控器在得知这一信息后,可以对整个客户虚拟机的内存空间进行转储(dump),并在事后进行查找分析,导致用户的敏感隐私数据极有可能因此而被暴露偷窥。

尽管恶意虚拟机监控器起先获取的可能只是庞杂的原始字节数据,其可以利用特殊寄存器数据、特定操作系统内存地址空间结构等信息作为提示,借助虚拟机自省(Virtual Machine Introspection,VMI)等技术手段,从中萃取出隐私数据和语义信息,毕竟这在理论上是可能的。

\subsection{客户虚拟机的磁盘安全}

在真实机器上读磁盘的过程可以简化为此,操作系统先在内存中预留一段区域,然后将待读取数据在磁盘上的位置和预留内存地址等信息通过IO指令告诉磁盘,磁盘通过直接内存访问(Direct Memory Access,DMA)将数据填写到预留内存区域,最后发送中断告诉操作系统数据已经准备就绪。写磁盘的过程基本与此类似。在虚拟化环境下,虚拟机监控器模拟了上述DMA的过程,其在根模式截获敏感IO指令,从磁盘映像中获取对应数据并填至客户虚拟机内存,最后注入虚拟中断告知完成。可以看到,虚拟机监控器处在客户虚拟机的IO数据路径之中,其可以对磁盘IO数据进行任意偷窥甚至修改,威胁客户虚拟机的磁盘安全。

\section{Secure KVM安全嵌套式虚拟化系统的设计与实现}

\subsection{系统总体架构}

\subsection{客户虚拟机内存安全的保证}

\subsubsection{基本原理}

\subsubsection{特殊边界情形}

\subsection{客户虚拟机磁盘安全的保证}

\subsubsection{基本原理}

\subsubsection{特殊边界情形}



\section{实验与性能测试}



\section{相关研究工作分析}



\section{总结与展望}
